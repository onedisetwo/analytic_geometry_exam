\section{Определение канонического уравнения кривой второй степени через значения инвариантов и полуинварианта}

\begin{theorem}
    Следующая таблица даёт необходимые и достаточные условия принадлежности кривой второго порядка к одному из девяти видов в терминах инвариантов.

    \begin{center}
        \begin{tabular}{| l | l l |}
            \hline
            \textit{Эллипс} & $I_2 > 0$, & $I_1I_3 < 0$\\
            \textit{Мнимый эллипс} & $I_2 > 0$, & $I_1I_3 > 0$\\
            \textit{Пара мнимых пересекающихся прямых} & $I_2 > 0$, & $I_3 = 0$\\
            \textit{Гипербола} & $I_2 < 0$, & $I_3 \ne 0$\\
            \textit{Пара пересекающихся прямых} & $I_2 < 0$, & $I_3 = 0$\\
            \textit{Парабола} & $I_2 = 0$, & $I_3 \ne 0$\\
            \textit{Пара параллельных прямых} & $I_2 = I_3 = 0$, & $I_2^\ast < 0$\\
            \textit{Пара мнимых параллельных прямых} & $I_2 = I_3 = 0$, & $I_2^\ast > 0$\\
            \textit{Пара совпавших прямых} & $I_2 = I_3 = 0$, & $I_2^\ast = 0$\\
            \hline
        \end{tabular}
    \end{center}
\end{theorem}

\begin{proof}
    Проведём доказательство для эллипса (для остальных аналогично). 

    $\Rightarrow$. Каноническое уравнение имеет вид
    $$
    \frac{x^2}{a^2} + \frac{y^2}{b^2} - 1 = 0.
    $$
    Для него вычисляем значения инвариантов:
    $$
    I_1 = \frac{1}{a^2} + \frac{a}{b^2} > 0,\quad I_2 = \frac{1}{a^2}\cdot\frac{1}{b^2} > 0,\quad I_3 = -\frac{1}{a^2}\cdot\frac{1}{b^2} < 0.
    $$
    Итак, $I_2 > 0$ и $I_1I_3 < 0$. При умножении на константу $\lambda \ne 0$ инвариант $I_2$ умножается на $\lambda^2$ и не меняет знак, а $I_1I_3$ умножается на $\lambda^4$ и тоже не меняет знак.

    $\Leftarrow$. Пусть имеем уравнение второго порядка с указанными инвариантами. Приведём матрицу квадратичной части к диагональному виду (при этом инварианты не изменяются, т.\,к. мы не домножаем ни на какую константу, мы только делаем поворот), а затем сделаем сдвиг (тоже ни на что не домножаем). Тогда получим матрицы
    $$
    Q = 
    \begin{pmatrix}
        a_{11} & 0\\
        0 & a_{22}
    \end{pmatrix},\quad\mathcal{A} = 
    \begin{pmatrix}
        a_{11} & 0 & 0\\
        0 & a_{22} & 0\\
        0 & 0 & a_0
    \end{pmatrix}
    $$
    Мы знаем про инварианты:
    $$
    I_2 = a_{11}a_{22} > 0,\quad I_1I_3 = (a_{11} + a_{22})a_{0} < 0.
    $$
    $a_{11}a_{22} > 0$, значит, либо оба коэффициента больше нуля, либо оба меньше. Если оба больше, то их сумма положительная и $a_0 < 0$ (из второго неравенства). Перенеся $a_0$ в правую часть и разделив на $-a_0$, получим действительно уравнение эллипса:
    $$
    \underbrace{(a_{11} / (-a_0))}_{1/a^2}x^2 + \underbrace{(a_{22} / (-a_0))}_{1/b^2}y^2 = 1.
    $$
    Если оба коэффициента меньше нуля, то их сумма меньше нуля и $a_0 > 0$ и, сделав те же операции, что и в прошлом случае, получаем уравнение эллипса.
\end{proof}


