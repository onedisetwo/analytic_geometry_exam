\section{Задание кривой второй степени в полярных координатах. Рациональная параметризация кривой второго порядка}

\begin{theorem}[Директориальное свойство коник]
    Пусть $F$ --- некоторая точка, $d$ --- некоторая прямая, такая что $F \notin d$, $e$ --- некоторое положительное число. Тогда геометрическое место точек $X$, таких что
    $$
    |XF| = e\rho(X, d)
    $$
    является
    \begin{itemize}
        \item эллипсом, при $e < 1$
        \item гиперболой, при $e > 1$
        \item параболой, при $e = 1$
    \end{itemize}
\end{theorem}

\begin{proof}
    Поместим фокус в начало системы координат, а директрису расположим параллельно оси ординат на расстоянии $s$. При $e = 1$, очевидно, что данное ГМТ --- парабола.

    Рассмотрим случай, когда $e \ne 1$.
    $$
    |XF| = \sqrt{x^2 + y^2} = e|x + s|.
    $$
    Возведём в квадрат:
    $$
    \begin{array}{c}\displaystyle
        x^2 + y^2 = e^2\cdot x^2 + 2e^2s\cdot x + e^2\cdot s^2\\\displaystyle
        (1 - e^2)x^2 - 2e^2s\cdot x + y^2 = e^2s^2\\\displaystyle
        %в следующей строке забыл возвести в квадрат скобку с x, не уверен насчёт синтаксиса, так что коммент
        (1 - e^2)\left(x - \frac{e^2s}{1 - e^2}\right) + y^2 = e^2s^2 + \frac{e^4s^2}{1 - e^2} = \frac{e^2s^2}{1 - e^2}.
    \end{array}
    $$
    Сделаем замену координат:
    $$
    \begin{cases}\displaystyle
        \widehat{x} = x - \frac{e^2s}{1 - e^2},\\\displaystyle
        \widehat{y} = y.
    \end{cases}
    $$
    Получим уравнение:
    $$
    \frac{(1 - e^2)^2}{e^2s^2}x^2 + \frac{(1 - e^2)}{e^2s^2}y^2 = 1.
    $$
    При $e < 1$ получим уравнение эллипса, при $e > 1$ --- гиперболы.
\end{proof}

\begin{definition}
    Число $e$ называется \textbf{эксцентриситетом}. Число $p = es$ называется \textbf{фокальным параметром}.
\end{definition}

Формулы, связывающие все величины, которые у нас появились:
$$
p = \frac{b^2}{a},\quad e = \frac{c}{a}.
$$
Также
$$
c = \sqrt{a^2 - b^2}\text{ для эллипса},\quad c = \sqrt{a^2 + b^2}\text{ для гиперболы}.
$$

\begin{theorem}
    В обобщённых полярных координатах коники пишутся так:
    $$
    r = \frac{p}{1 - e\cos\varphi}.
    $$
\end{theorem}

\begin{proof}
    В обобщённых полярных координатах одну и ту же точку можно задать как координатами $(r, \varphi)$, так и координатами $(-r, \varphi + \pi)$. Поэтому в записи директориального свойства в полярных координатах
    $$
    r = e|r\cos\varphi + s|
    $$
    можно раскрыть модуль без дополнительных знаков. Приведя к нормальному виду, получаем
    $$
    r = \frac{p}{1 - e\cos\varphi}.
    $$
\end{proof}


