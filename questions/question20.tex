\section{Алгебраические кривые на плоскости. Теорема <<об отщеплении прямой>>}

\begin{definition}
    Кривая называется \textbf{алгебраической}, если в некоторой аффинной системе координат она задаётся уравнением
    $$f(x_1, \ldots, x_k) = 0,$$
    где $f$ --- многочлен. \textbf{Порядком} кривой называется степень $f$.
\end{definition}

\begin{theorem}[Об отщеплении прямой]
    Алгебраическая кривая произвольного порядка на плоскости содержит в себе прямую $\ell: \underbrace{Ax + By + C}_{f} = 0$ тогда и только тогда, когда $f \mid F$.
\end{theorem}

\begin{proof}
    $\Leftarrow$. Тогда множество решений уравнения $f = 0$ входит в множество решений $F = 0$ (иными словами, $f = 0 \Rightarrow F = 0$), значит, в ГМТ: $F = 0$ содержится ГМТ: $f = 0$, т.\,е. данная прямая.

    $\Rightarrow$. Не ограничивая общности, пусть $A \ne 0$. Тогда разделим $F$ на $f$ как многочлены от $x$ с остатком $r(y)$, т.\,е. получим разложение $F(x) = F_1(x)\cdot f(x) + r(y)$. предположим, что $r \ne 0$, т.\,е. найдётся точка $y_0$, в которой $r(y_0) \ne 0$. Выберем $x_0$ так, чтобы выполнялось $f(x_0, y_0) = 0$: $x_0 = -1/A(By_0 + C)$. Тогда
    $$0 = F(x_0, y_0) = f(x_0, y_0) \cdot F_1(x_0, y_0) + r(y_0) = r(y_0).$$
    Полученное противоречие доказывает утверждение теоремы.
\end{proof}

\begin{theorem}
    Пусть имеем алгебраическую кривую $\Gamma: f = 0$. При некоторой аффинной замене координат она будет иметь уравнение $\Gamma: g = 0$. Тогда $g$ --- многочлен, причём $\deg g = \deg f$.
\end{theorem}

\begin{proof}
    Сначала докажем, что $g$ --- многочлен. Для этого вспомним формулу аффинного преобразования координат:
    $$
    \begin{pmatrix}
        x_1\\
        \vdots\\
        x_k
    \end{pmatrix} = C
    \begin{pmatrix}
        x_1^\prime\\
        \vdots\\
        x_k^\prime
    \end{pmatrix} + O^\prime.
    $$
    Выразив $x_1, \ldots, x_k$ из неё и подставив в уравнение $f = 0$, получим уравнение $\Gamma$ в новой системе координат. Оно, очевидно, является многочленом.

    Теперь покажем, что $\deg g = \deg f$. Во-первых, $\deg f \geqslant \deg g$ (т.\,к. выражения старых координат через новые линейно, степень не могла увеличиться). А во-вторых, $\deg g \geqslant \deg f$, потому что можно прийти от $g$ к $f$ обратным преобразованием (матрица $C$ обратима). Отсюда следует $\deg f = \deg g$.
\end{proof}


