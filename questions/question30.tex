\section{Парабола и её геометрические свойства}

\begin{definition}[Геометрическое определение параболы]
    \textbf{Парабола} --- геометрическое место точек $X$, равноудалённых от фиксированной точки $F$ и фиксированной прямой $d$. Точка $F$ называется \textbf{фокусом} параболы, а прямая $d$ --- её \textbf{директрисой}.
\end{definition}

\begin{statement}
    Аналитическое и геометрическое определения параболы эквивалентны.
\end{statement}

\begin{theorem}[Оптическое свойство]
    Касательная в точке $X$ параболы является биссектрисой угла $\angle F_1XH$, где $H$ --- основание перпендикуляра из $X$ на $d$.
\end{theorem}

\begin{remark}
    Аналога изогонального свойства для параболы нет.
\end{remark}

