\section{Прямолинейные образующие поверхностей второй степени}

\begin{definition}
    Назовём \textbf{прямолинейной образующей} поверхности прямую, целиком в ней содержащуюся. Как правило, это понятие не применяется к распадающимся поверхностям.
\end{definition}

Прямолинейные образующие есть только у тех кривых, у которых в уравнениях выделяются разности квадратов. Это 
$$
\underset{
\begin{array}{c}
    \text{\footnotesize однополостный}\\ \text{\footnotesize гиперболоид}
\end{array}}{\frac{x^2}{a^2} - \frac{z^2}{c^2} = 1 - \frac{y^2}{b^2}},\quad\underset{
\begin{array}{c}
    \text{\footnotesize гиперболический}\\ \text{\footnotesize параболоид}
\end{array}}{\frac{x^2}{p} - \frac{y^2}{q} = 2z},\quad\underset{
\begin{array}{c}
    \text{\footnotesize конус}
\end{array}}{\frac{x^2}{a^2} = \frac{z^2}{c^2} - \frac{y^2}{b^2}}.
$$

\begin{theorem}
    Прямолинейные образующие любой поверхности имеют асимптотическое направление.
\end{theorem}

\begin{proof}
    Прямая с этим направлением (по определению) содержится в поверхности, а если её направление неасимптотическое, то она либо её не пересекает, либо пересекает в двух точках (быть может, совпавших). Доказательство этих утверждений такое же, как и в плоскости.
\end{proof}

Утверждение ниже (с почти неизменным доказательством) будет верно и для однополостного гиперболоида (а конус нам неинтересен).

\begin{theorem}
    Гиперболический параболоид имеет два семейства прямолинейных образующих. Через каждую точку проходит ровно одна прямая каждого семейства, и эти две прямые пересекаются ровно по этой точке. Две различные прямые из одного семейства скрещиваются, а из разных --- пересекаются.
\end{theorem}

\begin{proof}
    Асимптотические направления $(\alpha, \beta, \gamma)$ гиперболического параболоида находятся из уравнения: $\displaystyle\frac{\alpha^2}{p} - \frac{\beta^2}{q} = 0$, т.\,е. лежат в плоскостях
    $$
    \pi_1: \frac{\alpha}{\sqrt{p}} - \frac{\beta}{\sqrt{q}} = 0,\quad \pi_2: \frac{\alpha}{\sqrt{p}} + \frac{\beta}{\sqrt{q}} = 0.
    $$
\end{proof}


