\section{Полуинвариант кривой второй степени}

\begin{definition}
    $$\displaystyle I_2^\ast \vcentcolon = \det
    \begin{pmatrix}
        a_{11} & a_1\\
        a_1 & a_0
    \end{pmatrix} + 
    \begin{pmatrix}
        a_{22} & a_2\\
        a_2 & a_0
    \end{pmatrix}
    $$
    называется \textbf{полуинвариантом} (или <<почти инвариантом>>, как у Веселова и Троицкого) уравнения второго порядка.
\end{definition}

\begin{theorem}
    Если $I_2 = I_3 = 0$, то $I_2^\ast$ является ортогональным инвариантом\footnotemark.
\end{theorem}

\footnotetext{Ещё $I_2^\ast$ сохраняется независимо от значений остальных инвариантов при ортогональных преобразованиях с общим началом координат.}

Для начала, докажем вспомогательную лемму

\begin{lemma}
    Характеристический многочлен матрицы $\mathcal{A}$ инвариантен относительно прямоугольных замен координат с общим началом.
\end{lemma}

\begin{proof}
    В этом случае имеем $x_0 = y_0 = 0$ и 
    $$
    D = 
    \begin{pmatrix}
        \multicolumn{2}{c}{\multirow{2}{*}{$C$}} & 0\\
        \multicolumn{2}{c}{} & 0\\
        0 & 0 & 1
    \end{pmatrix}
    $$
    Заметим, что написана ортогональная матрица, а значит, можем повторить доказательство теоремы 24.1.
\end{proof}

\begin{proof}
    Характеристический многочлен матрицы $\mathcal{A}$ имеет вид
    $$
    \begin{array}{c}\displaystyle
        \det
        \begin{pmatrix}
            a_{11} - \lambda & a_{12} & a_1\\
            a_{12} & a_{22} - \lambda & a_2\\
            a_1 & a_2 & a_0 - \lambda
        \end{pmatrix} = 
        -\lambda^3 + (a_0 + a_{11} + a_{22})\lambda^2 - {}\\\displaystyle {} - \left(
            \det
            \begin{pmatrix}
                a_{11} & a_1\\
                a_1 & a_0
            \end{pmatrix} + \det
            \begin{pmatrix}
                a_{22} & a_2\\
                a_2 & a_0
            \end{pmatrix} + \det
            \begin{pmatrix}
                a_{11} & a_{12}\\
                a_{12} & a_{22}
            \end{pmatrix}
        \right)\lambda + \det \mathcal{A} = {}\\\displaystyle {} = 
        -\lambda^3 + (a_0 + I_1)\lambda^2 - (I_2^\ast + I_2)\lambda + I_3.
    \end{array}
    $$
    Если производится замена с тем же началом, то (в силу предыдущей леммы) коэффициенты (а значит, и $I_2^\ast$) сохраняются. Поэтому осталось показать инвариантность $I_2^\ast$ при сдвигах. Тут уже нам понадобится условие равенства нулю $I_2$ и $I_3$. Во-первых, можно считать $a_{12}$ равным нулю, т.\,к. (мы доказывали в ортогональной классификации кривых второго порядка) это можно сделать одним поворотом. Поэтому из $I_2 = 0$ получаем $a_{11}a_{22} = 0$. Без ограничения общности, можно положить $a_{22} = 0$ (<<поменять местами $x$ и $y$>> --- это ортогональная замена, сохраняющая начало координат). Из соотношения $I_3 = -a_2^2a_{11} = 0$ получаем $a_2 = 0$ (получить $I_3$ легко, разложив его по второй строке; $a_{11} \ne 0$, так как иначе уменьшается степень уравнения). Рассмотрим сдвиг
    $$
    x = \widetilde{x} + x_0,\quad y = \widetilde{y} + y_0.
    $$
    Тогда
    $$
    \widetilde{F} = a_{11}(\widetilde{x} + x_0)^2 + 2a_1(\widetilde{x} + x_0) + a_0 = \underbrace{a_{11}}_{\widetilde{a}_{11}}\widetilde{x}^2 + 2\underbrace{(a_{11}x_0 + a_1)}_{\widetilde{a}_{1}}\widetilde{x} + \underbrace{(a_{11}x_0^2 + 2a_1x_0 + a_0)}_{\widetilde{a}_0},
    $$
    при этом 
    $$
    \mathcal{A} =
    \begin{pmatrix}
        a_{11} & 0 & a_1\\
        0 & 0 & 0\\
        a_1 & 0 & a_0
    \end{pmatrix},
    $$
    поэтому $I_2^\ast = a_{11}a_0 - a_1^2$. Отсюда находим
    $$
    \widetilde{I}_2^\ast = \widetilde{a}_{11}\widetilde{a}_0 - \widetilde{a}_1^2 = a_{11}(a_{11}x_0^2 + 2a_1x_0 + a_0) - (a_{11}x_0 + a_1)^2 = a_{11}a_0 - a_1^2 = I_2^\ast.
    $$
\end{proof}


