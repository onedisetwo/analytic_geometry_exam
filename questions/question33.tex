\section{Поверхности второй степени. Аффинная классификация}

Тут всё то же самое, что и для кривых за исключением доказательства того, что если поверхности имеют разные названия, то они аффинно неэквивалентны.

\begin{lemma}
    $r \vcentcolon= \rk Q$ и $R \vcentcolon= \rk \mathcal{A}$ являются аффинными инвариантами.
\end{lemma}

\begin{proof}
    В курсе алгебры доказывалась верхняя и нижняя оценка на ранг матрицы. Выглядело это так:
    $$
    \rk A + \rk B - n \leqslant \rk AB \leqslant \min\{\rk A, \rk B\}.
    $$
    По лемме 22.1 мы знаем, что $Q^\ast = C^TQT$ (там нигде не использовалось, что матрица ортогональна), отсюда можно получить, что
    $$
    \rk Q \leqslant \rk Q^\ast \leqslant \rk Q,
    $$
    аналогично для матрицы $\mathcal{A}$.
\end{proof}

\begin{orangebox}
    Делая выводы из предыдущей леммы, можно разделить поверхности на классы следующим образом:
    \begin{center}
        \begin{tabular}{| c | l | l |}
            \hline
            & \textbf{Условия} & \textbf{Поверхности}\\
            \hline
            \hline
            \rnum{1} & $r = 3$, $R = 3$ или $R = 4$ & 1 "---6\\
            \hline
            \rnum{2} & $r = 2$, $R = 4$ & 7, 8\\
            \hline
            \rnum{3} & $r = 2$, $R = 2$ или $R = 3$ & 9 "---13\\
            \hline
            \rnum{4} & $r = 1$, $R = 3$ & 14\\
            \hline
            \rnum{4} & $r = 1$, $R = 1$ или $R = 2$ & 15 "---17\\
            \hline
        \end{tabular}
    \end{center}
\end{orangebox}

Из того, что $r$ и $R$ являются аффинными инвариантами, достаточно показать неэквивалентность в пределах каждого из классов \rnum{1} "---\rnum{5}.

Точка (мнимый конус), прямая (пара мнимых пересекающихся плоскостей), пары параллельных, пересекающихся или совпавших плоскостей, очевидно, неэквивалентны ни друг другу, ни другим поверхностям. Рассмотрим пустые множества: мнимый эллиптический цилиндр имеет одно асимптотическое направление, мнимый эллипсоид не имеет асимптотических направлений, пара мнимых параллельных плоскостей имеет целую плоскость асимптотических направлений. Кроме того, эллипсоид ограничен, в толичие от других нерассмотренных поверхностей. Для оставшихся типов имеет следующие таблицы (первая --- для типа \rnum{1}, вторая --- для \rnum{3}):

\begin{center}
    \begin{tabular}{| c | c | c |}
        \hline
        \textbf{Название} & \textbf{Наличие центров} & \textbf{Прямолинейные образующие}\\
        \hline\hline
        однополостный гиперболоид & 1 & есть\\
        \hline
        двуполостный гиперболоид & 1 & нет\\
        \hline
        эллиптический параболоид & нет & нет\\
        \hline
        гиперболический параболоид & нет & есть\\
        \hline
    \end{tabular}
\end{center}

\begin{center}
    \begin{tabular}{| c | c | c |}
        \hline
        \textbf{Название} & \textbf{Наличие центров} & \textbf{Прямолинейные образующие}\\
        \hline\hline
        эллиптический цилиндр & прямая & одно\\
        \hline
        гиперболический цилиндр & прямая & две плоскости\\
        \hline
        параболический цилиндр & нет & неважно\footnotemark\\
        \hline
    \end{tabular}
\end{center}

\footnotetext{Действительно, неважно, потому что нам нужно лишь показать отличие параболического цилиндра от остальных. Однако, если интересно, асимптотические направления параболического цилиндра --- это все, параллельные плоскости $y = 0$.}


