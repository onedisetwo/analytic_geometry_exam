\section{Прямая на плоскости. Параметрическое задание и задание уравнением}

\begin{definition}[от Сабира Меджидовича]
    \textbf{Прямая} --- траектория равномерно движущейся частицы.
\end{definition}

Следуя этому определению, можно сказать, что прямую на плоскости однозначно задаёт начальная точка движения $(x_0, y_0)$ и вектор скорости частицы $(\alpha, \beta) \ne (0, 0)$. Исходя из этих соображений, можем записать \textbf{параметрическое задание прямой}:
$$
\begin{cases}
    x = x_0 + \alpha t,\\
    y = y_0 + \beta t
\end{cases}\quad t \in \R.
$$

По сути, мы выбираем на прямой точку и направляющий вектор. А вместе мы называли это <<репер>>, на нём $t$ --- аффинная координата. Поэтому на самом деле, мы не только задаём объект, но и выбираем на нём систему координат. И то же самое будет с плоскостями в пространстве.

\begin{statement}
    Пусть имеем две прямые $\ell_1 = \{P_1 + t\cdot v_1\}_{t \in \R}$ и $\ell_2 = \{P_2 + t\cdot v_2\}_{t \in \R}$. Тогда
    $$\ell_1 = \ell_2 \Leftrightarrow \rk\left\{v_1, v_2, \overrightarrow{P_1P_2}\right\} = 1.$$
\end{statement}

\begin{proof}
    Условие равносильно тому, что система
    $$P_1 + t \cdot v_1 = P_2 + s \cdot v_2$$
    имеет бесконечно много решений. Записывая её в матричном виде, получим
    $$
    \begin{pmatrix}
        v_1 & -v_2 & \bord & \overrightarrow{P_1P_2}
    \end{pmatrix}.
    $$

    Условие на совместность (теорема Кронекера "---Капеллли):
    $$\rk
    \begin{pmatrix}
        v_1 & -v_2
    \end{pmatrix} = \rk
    \begin{pmatrix}
        v_1 & -v_2 & \bord & \overrightarrow{P_1P_2}
    \end{pmatrix},
    $$

    условие на неопределённость
    $\rk
    \begin{pmatrix}
        v_1 & -v_2
    \end{pmatrix} < 2
    $,
    ну и ясно дело ранг этой матрицы не может равняться нулю (иначе оба вектора нулевые). Отсюда и получаем требуемое.
\end{proof}

Прямую можно задать не только параметрически, но и уравнением первого порядка:
$$
Ax + By + C = 0,\quad(A, B) \ne (0, 0).
$$

\begin{theorem}
    Два этих способа задания эквивалентны.
\end{theorem}

\begin{proof}
    $\Rightarrow$. Пусть имеем параметрическое задание прямой $\ell$:
    $$
    \begin{cases}
        x = x_0 + \alpha t,\\
        y = y_0 + \beta t.
    \end{cases} \Rightarrow 
    \begin{cases}
        -\beta x = -\beta x_0 -\alpha\beta t,\\
        \alpha y = \alpha y_0 + \alpha\beta t.
    \end{cases}
    $$
    Сложив получившиется уравнения, получим уравнение первого порядка, в котором коэффициенты при $x$ и при $y$ не равны нулю одновременно.

    $\Leftarrow$. Заметим, что при $(A, B) \ne (0, 0)$ уравнение $Ax + By + C = 0$ имеет хотя бы одно решение (из теоремы Кронекера "---Капелли). Обозначим его за $(x_0, y_0)$. Тогда можем выписать семейство решений уравнения
    $$
    \begin{cases}
        x = x_0 - Bt,\\
        y = y_0 + At
    \end{cases}\quad t \in \R.
    $$

    Получили систему уравнений, в которой коэффициенты при $t$ не равны нулю одновременно.
\end{proof}

\begin{remark}
    Из доказательства этого утверждения мы получили, что направляющий вектор прямой $Ax + By + C = 0$ можно выписать в виде $(-B, A)$. Как следствие, \textit{любой направляющий вектор прямой должен обнулять её линейную часть} (причём, это критерий).
\end{remark}

\begin{statement}[может быть полезно для решения задач]
    Пусть имеем две прямые $\ell_1 = (x_0, y_0) + (\alpha, \beta)t$ и $\ell_2: Ax + By + C = 0$. Тогда
    $$
    \ell_1 = \ell_2 \Leftrightarrow
    \begin{cases}
        Ax_0 + By_0 + C = 0,\\
        A\alpha + B\beta = 0.
    \end{cases}
    $$
\end{statement}

\begin{proof}
    Первое условие равносильно $(x_0, y_0) \in \ell_2$, а второе --- что $(\alpha, \beta) \parallel \ell_2$. Теперь утверждение становится очевидным.
\end{proof}

\begin{statement}
    Пусть имеем две прямые $\ell_1: A_1x + B_1y + C_1 = 0$ и $\ell_2: A_2x + B_2y + C_2 = 0$. Тогда
    $$
    \ell_1 = \ell_2 \Leftrightarrow \frac{A_1}{A_2} = \frac{B_1}{B_2} = \frac{C_1}{C_2}.
    $$
    Здесь деление подразумевается в смысле пропорциональности: знаменатель может быть равен нулю, но тогда равен нулю и числитель.
\end{statement}

\begin{proof}
    Справедливость утверждения в сторону $\Leftarrow$ очевидна. Здесь докажем в другую сторону. Запишем условие совпадения прямых алгебраически --- система
    $$
    \begin{cases}
        A_1x + B_1y + C_1 = 0,\\
        A_2x + B_2y + C_2 = 0
    \end{cases}
    $$
    имеет бесконечно много решений. Это значит, что 
    $$
    \rk
    \begin{pmatrix}
        A_1 & B_1\\
        A_2 & B_2
    \end{pmatrix} = 
    \rk
    \begin{pmatrix}
        A_1 & B_1 & C_1\\
        A_2 & B_2 & C_2
    \end{pmatrix} = 1.
    $$

    Отсюда и следует требуемое.
\end{proof}

