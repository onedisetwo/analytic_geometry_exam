\section{Инварианты поверхности второй степени. Частичная классификация поверхностей второй степени с помощью инвариантов}

Нам эту задачу давал на дом Александр Александрович (задачи 720 "---722 в задачнике Ю.\,М. Смирнова) и в тот вечер я не пожалел времени на аналитическую геометрию, поэтому здесь будет полная классификация поверхностей. На экзамене можно рассказать всё до полуинвариантов (их спрашивать не должны), но, если хотите удивить экзаменатора, расскажите про полуинварианты тоже, ему понравится.

Здесь (как и в случае поверхностей) ортогональными инвариантами будут являться коэффициенты характеристического многочлена матрицы квадратичной формы и определитель <<большой>> матрицы поверхности. Характеристический многочлен выглядит теперь так:

$$
\chi(\lambda) = \det(Q - \lambda E) = -\lambda^3 + I_1\lambda^2 - I_2\lambda_3 + I_3,
$$
где
$$
\begin{array}{c}\displaystyle
    I_1 \vcentcolon = \tr Q = a_{11} + a_{22} + a_{33},\quad I_2 \vcentcolon = \det
    \begin{pmatrix}
        a_{11} & a_{12}\\
        a_{12} & a_{22}
    \end{pmatrix} + \det
    \begin{pmatrix}
        a_{22} & a_{23}\\
        a_{23} & a_{33}
    \end{pmatrix} + \det
    \begin{pmatrix}
        a_{11} & a_{13}\\
        a_{13} & a_{33}
    \end{pmatrix},\\\displaystyle
    I_3 \vcentcolon = \det Q = \det
    \begin{pmatrix}
        a_{11} & a_{12} & a_{13}\\
        a_{12} & a_{22} & a_{23}\\
        a_{13} & a_{23} & a_{33}
    \end{pmatrix},\quad I_4 \vcentcolon = \det \mathcal{A} = 
    \det
    \begin{pmatrix}
        a_{11} & a_{12} & a_{13} & a_1\\
        a_{12} & a_{22} & a_{23} & a_2\\
        a_{13} & a_{23} & a_{33} & a_3\\
        a_1 & a_2 & a_3 & a_0
    \end{pmatrix}.
\end{array}
$$

А ещё здесь есть полуинварианты, которые являются инвариантами при заменах координат с общим началом:
$$
\begin{array}{c}\displaystyle
    I_3^\ast = \det
    \begin{pmatrix}
        a_{11} & a_{12} & a_1\\
        a_{12} & a_{22} & a_2\\
        a_1 & a_2 & a_0
    \end{pmatrix} + \det
    \begin{pmatrix}
        a_{22} & a_{23} & a_2\\
        a_{23} & a_{33} & a_3\\
        a_2 & a_3 & a_0
    \end{pmatrix} + \det
    \begin{pmatrix}
        a_{11} & a_{13} & a_1\\
        a_{13} & a_{33} & a_3\\
        a_1 & a_3 & a_0
    \end{pmatrix},\\\displaystyle I_2^\ast = \det
    \begin{pmatrix}
        a_{11} & a_1\\
        a_1 & a_0
    \end{pmatrix} + \det
    \begin{pmatrix}
        a_{22} & a_2\\
        a_2 & a_0
    \end{pmatrix} + \det
    \begin{pmatrix}
        a_{33} & a_3\\
        a_3 & a_0
    \end{pmatrix}.
\end{array}
$$

Доказывается это так же, как и в двумерном случае --- сначала доказываем аналог леммы 25.1, а потом так же (втупую, через арифметику и боль) доказываем аналоги теоремы 25.1:

\begin{theorem}
    $I_3^\ast$ является ортогональным инвариантом, если $I_3 = I_4 = 0$.
\end{theorem}

\begin{theorem}
    $I_2^\ast$ является ортогональным инвариантом, если $I_2 = I_3 = I_4 = I_3^\ast = 0$.
\end{theorem}

Итак, классификация:

\begin{theorem}
    Следующая таблица даёт необходимые и достаточные условия принадлежности кривой второго порядка к одному из семнадцати видов в терминах инвариантов:
    \begin{tabular}{| l | l |}
        \hline
        \textit{Эллипсоид} & $I_2 > 0$, $I_1I_3 > 0$, $I_4 < 0$\\
        \textit{Мнимый эллипсоид} & $I_2 > 0$, $I_1I_3 > 0$, $I_4 > 0$\\
        \textit{Мнимый конус} & $I_2 > 0$, $I_1I_3 > 0$, $I_4 = 0$\\
        \textit{Однополостный гиперболоид} & $I_3 \ne 0$, $I_4 > 0$ и $(I_2 \leqslant 0) \vee I_1I_3 \leqslant 0$\\
        \textit{Двуполостный гиперболоид} & $I_3 \ne 0$, $I_4 < 0$ и $(I_2 \leqslant 0) \vee I_1I_3 \leqslant 0$\\
        \textit{Конус} & $I_2 \leqslant 0$ или $I_1I_3 \leqslant 0$, $I_4 = 0$\\
        \textit{Эллиптический параболоид} & $I_3 = 0$, $I_4 < 0$\\
        \textit{Гиперболический параболоид} & $I_3 = 0$, $I_4 > 0$\\
        \hline
        \textit{Эллиптический цилиндр} & $I_4 = I_3 = 0$, $I_2 > 0$, $I_1I_3^\ast < 0$\\
        \textit{Мнимый эллиптический цилиндр} & $I_4 = I_3 = 0$, $I_2 > 0$, $I_1I_3^\ast > 0$\\
        \textit{Пара мнимых пересекающихся плоскостей} & $I_4 = I_3 = I_3^\ast = 0, I_2 > 0$\\
        \textit{Гиперболический цилиндр} & $I_4 = I_3 = 0$, $I_2 < 0$, $I_3^\ast \ne 0$\\
        \textit{Пара действительных пересекающихся плоскостей} & $I_4 = I_3 = I_3^\ast = 0$, $I_2 < 0$\\
        \textit{Параболический цилиндр} & $I_4 = I_3 = I_2 = 0$, $I_3^\ast \ne 0$\\
        \textit{Пара действительных параллельных плоскостей} & $I_4 = I_3 = I_2 = I_3^\ast = 0$, $I_2^\ast < 0$\\
        \textit{Пара мнимых параллельных плоскостей} & $I_4 = I_3 = I_2 = I_3^\ast = 0$, $I_2^\ast > 0$\\
        \textit{Пара совпавших плоскостей} & $I_4 = I_3 = I_2 = I_3^\ast = I_2^\ast = 0$\\
        \hline
    \end{tabular}
\end{theorem}

\begin{proof}
    Всё, что до черты --- это то, что вас обязаны спросить на экзамене, это знать надо. После черты --- полуинварианты, их можно не знать. Я умею доказывать красиво всё, что до черты, после черты мы поступаем аналогично случаю для кривых (там разобран только эллипс, но в остальных случаях и даже здесь всё правда аналогично). 

    Итак, как же доказывать это красиво. А вот как. У всего, что мы умеем классифицировать без полуинварианта есть нечто общее --- у них нет линейной части (только квадратичная и свободный коэффициент), поэтому их однозначно определяет знак коэффициентов квадратичной части и знак свободного члена. А коэффициенты квадратичной части --- это корни характеристического уравнения и все наши требования к нему будут выражаться в формулировках вроде <<у этого уравнения должно быть 3 корня одного знака>> (кстати, это будет эллипсоид или мнимый эллипсоид). А это уже вполне алгебраическое условие на коэффициенты многочлена. А эти коэффициенты как раз являются ортогональными инвариантами $I_1$, $I_2$ и $I_3$. Потом нам нужно согласовать знак корней со знаком свободного члена. Например, в случае эллипсоида три корня должны быть одного знака, а свободный коэффициент --- другого. Это выражается в условии $I_4 < 0$. Для мнимого эллипсоида --- наоборот, $I_4 > 0$, потому что теперь и свободный член должен быть такого же знака с корнями. Так разбираются все случаи. А всё, что с полуинвариантами, делается перебором.
\end{proof}


