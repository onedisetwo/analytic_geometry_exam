\section{Скалярное произведение векторов. Скалярное произведение в координатах (в ортогональных и в произвольных аффинных)}

\begin{definition}
    Функция $f(v)$ называется \textbf{линейной}, если $f(u + v) = f(u) + f(v)$ и $f(\lambda u) = \lambda f(u)$.
\end{definition}

По $f(e_1), \ldots, f(e_k)$ можно восстановить $f$:
$$f(v) = f\left(\sum_iv_ie_i\right) = \sum_iv_if(e_i).$$

Возникает система координат на множестве линейных функций: числа $f(e_1), \ldots, f(e_k)$ могут быть любыми и однозначно определяют $f$, поэтому вполне разумно называть их координатами функции $f$.

\begin{definition}[Скалярное произведение]
    $(u, v) = |u| \cdot |v| \cdot \cos\angle(u, v)$\hfill\textit{<<Пока так\ldots>>} (c)
\end{definition}

\begin{statement}[Свойства скалярного произведения\footnotemark]

    \begin{enumerate}[noitemsep, nolistsep]
        \item \textbf{Билинейность}: $(u + v, w) = (u, w) + (v, w)$, $(u, v + w) = (u, v) + (u, w)$, $(\lambda u, v) = \lambda (u, v) = (u, \lambda v)$.
        \item \textbf{Симметричность}: $(u, v) = (v, u)$.
        \item \textbf{Положительная определённость}: $u \ne 0 \Rightarrow (u, u) > 0$.
    \end{enumerate}

    Как сказал Иван Алексеевич, <<Свойства расположены в порядке убывания важности>>.
\end{statement}

\footnotetext{Потом станут аксиомами; все метрические теоремы геометрии выводятся из этих свойств}

\begin{remark}
    Для поиска проекции $v^\parallel$ вектора $v$ на вектор $u$ есть простая и полезная формула:
    $$v^\parallel = \frac{(u, v)}{(u, u)}u.$$
    Осознать её истинность можно, просто расписав скалярные произведения по определению. Из неё совсем легко выводятся два свойства ниже (без неё они тоже выводятся, но там нужно думать; с ней думать не нужно).
\end{remark}

\begin{proof}
    Докажем аддитивность (остальные пункты очевидны). Первое свойство --- $(u, v) = (u, v^\parallel)$, где $v^\parallel$ --- проекция вектора $v$ на направление вектора $u$. Второе свойство --- $(v + w)^\parallel = v^\parallel + w^\parallel$. Отсюда
    $$
    (u, v + w) = (u, (v + w)^\parallel) = (u, v^\parallel + w^\parallel) = (u, v^\parallel) + (u, w^\parallel) = (u, v) + (u, w).
    $$
    Линейность по второму аргументу доказывается аналогично.
\end{proof}

\begin{statement}
    Пусть имеем базис $\{e_1, \ldots, e_k\}$ и векторы $u = (u_1, \ldots, u_k)$ и $v = (v_1, \ldots, v_k)$ (координаты в данном базисе). Тогда
    $$(u, v) = \sum_{i, j = 1}^k u_iv_j(e_i, e_j). $$
\end{statement}

\begin{proof}
    Применим билинейность:
    $$
    (u, v) = \left(\sum_iu_ie_i, \sum_jv_je_j\right) = \sum_iu_i\left(e_i, \sum_jv_je_j\right) = \sum_iu_i\sum_jv_j(e_i, e_j) = \sum_{i, j = 1}^ku_iv_j(e_i, e_j).
    $$
    Получается, для подсчёта скалярного произведения нам нужно знать $k(k + 1) / 2$ чисел --- значения скалярного произведения на базисных векторах.
\end{proof}

\begin{definition}
    Базис $\{e_1, \ldots, e_k\}$ \textbf{ортонормирован}\footnotemark, если $(e_i, e_j) = \delta_{ij}$, где $\delta_{ij}$ --- символ Кронекера:
    $$\delta_{ij} = 
    \begin{cases}
        1,&\text{если}\;i = j\\
        0,&\text{если}\;i \ne j.
    \end{cases}$$
\end{definition}

\footnotetext{Отдельный вопрос --- <<А существуют ли вообще ортонормированные базисы?>>. Да, существуют. Самый простой пример --- стандартный базис в $\R^n$.}

\begin{remark}
    В ортонормированном базисе $(u, v) = u_1v_1 + \ldots + u_kv_k$.
\end{remark}

\begin{theorem}
    $f(\ast) = (u, \ast)$ --- общий вид линейной функции. Иными словами, любую линейную функцию от $\ast$ можно представить как скалярное произведение с каким-то вектором. При этом, $(u, \ast) = (v, \ast)$ в смысле равенства функций равносильно $u = v$.
\end{theorem}

\begin{proof}
    Заметим, что в теореме ни слова не сказано про базис, в котором мы работаем, а поэтому мы вольны выбрать какой нам удобно. А нам, конечно, удобно выбирать ортонормированный. Итак, пусть $\{e_1, \ldots, e_k\}$ --- ортонормированный базис, а $f$ --- какая-то линейная функция. Тогда она определяется значениями на базисных векторах. А можно сказать, что она определяется вектором $\mathcal{F} = \big(f(e_1), \ldots, f(e_k)\big)$. Тогда
    $$f(\ast) = \sum_i\ast_if(e_i) = \sum_i\ast_i\mathcal{F}_i = (\ast, \mathcal{F}).$$
    Вторая часть утверждения теперь очевидна.
\end{proof}

\begin{lemma}[из Веселова и Троицкого]
    Пусть в некотором ортонормированном базисе $\{e_1, \ldots, e_k\}$ вектор $a = (a_1, \ldots, a_k)$. Тогда
    $$a_i = (a, e_i),\quad i = 1, \ldots, k.$$
\end{lemma}

\begin{proof}
    Здесь я приведу 2 доказательства. Первое я сам придумал, и поэтому оно мне нравится, но второе намного проще.
    \begin{enumerate}
        \item Мы знаем, что координаты вектора определены единственным образом, поэтому осталось лишь доказать, что предложенные в формулировке подходят. Заметим, что функция $\displaystyle f(a) = \sum_{i = 1}^k(a, e_i)e_i$ линейна (из линейности скалярного произведения). В доказательстве предыдущей теоремы мы показали, что $f$ однозначно определяется вектором $\mathcal{F} = (f(e_1), \ldots, f(e_k))$. Подставим в $f$ вектор $e_j$ и посмотрим, что получится:
            $$f(e_j) = \sum_{i = 1}^k(e_j, e_i)e_i = (e_j, e_j)e_j = e_j,$$
            т.\,к. по определению ортонормированного базиса все скалярные произведения $(e_i, e_j)$ при $i \ne j$ зануляются. Значит, $\mathcal{F} = \{e_1, \ldots, e_k\}$, а $\displaystyle f(a) = (a, \mathcal{F}) = \sum_{i = 1}^ka_ie_i = a$. Получаем $\displaystyle\sum_{i = 1}^k(a, e_i)e_i = a$, то есть, $(a, e_i)$ подходят в качестве координат.
        \item А можно просто спроецировать вектор $a$ на базисные. Заметим, что из ортонормированности имеем
            $$
            a^{\parallel e_i} = \frac{(a, e_i)}{(e_i, e_i)}e_i = (a, e_i)e_i,
            $$
            Ясно, что $a$ есть сумма $a^{\parallel e_i}$ по всем $i$. А выписав сумму, мы получим буквально выражение $a$ через базисные векторы с коэффициентами $(a, e_i)$. Значит, это координаты (по определению). Единственный косяк --- я не понимаю, где мы использовали ортогональность базиса. Видимо, формула для проекции верна только в ортогональных базисах, но это нужно будет уточнить.
    \end{enumerate}
\end{proof}


