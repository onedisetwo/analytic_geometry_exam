\section{Ортогональные замены координат и ортогональные матрицы}

\begin{definition}
    Матрица $C$ \textbf{ортогональна}, если $CC^T = E$ (или же $C^T = C^{-1}$).
\end{definition}

Заметим, что при вычислении $CC^T$ мы умножаем строки матрицы $C$ на столбцы $C^T$, т.\,е. строки матрицы $C$. А в результате получаем $E$, что означает, что, перемножив $i$-ую строку матрицы $C$ на себя, мы получим $1$, а перемножив $i$-ую строку на $j$-ую, получим $0$. И то же самое со столбцами. И это не совпадение. Если текущий базис ортогональный, то в нём скалярные произведения пишутся как сумма произведений соответствующих координат векторов. А по столбцам в $C$ стоят как раз координаты нового базиса. А мы сейчас показали, что его матрица Грама единичная, а значит, он ортогональный. Ниже приведено другое доказательство этого же факта.

\begin{statement}
    Для ортогональной матрицы $C$ верно $\det C = \pm 1$.
\end{statement}

\begin{proof}
    $1 = \det E = \det(C^TC) = \det C^T \det C = (\det C)^2$.
\end{proof}

\begin{theorem}
    Пусть имеем два базиса $\{e_1, \ldots, e_k\}$ и $\{f_1, \ldots, f_k\}$, а $C$ --- матрица перехода от первого ко второму. Тогда из следующих утверждений любые два влекут третье:
    \begin{enumerate}[noitemsep, nolistsep]
        \item Базис $\{e_1, \ldots, e_k\}$ ортонормирован
        \item Базис $\{f_1, \ldots, f_k\}$ ортонормирован
        \item Матрица $C$ ортогональна
    \end{enumerate}
\end{theorem}

\begin{proof}
    Переформулируем условие. Что значит, что $\{e_1, \ldots, e_k\}$ ортонормирован? Это значит, что
    $$
    \begin{pmatrix}
        e_1 & \ldots & e_k
    \end{pmatrix}^T \cdot
    \begin{pmatrix}
        e_1 & \ldots & e_k
    \end{pmatrix} = E.
    $$
    Аналогично для $\{f_1, \ldots, f_k\}$. А что значит, что матрица $C$ ортогональна? Это значит, что $CC^T = E$. Теперь покажем, что из первого и третьего следует второе:
    $$
    \begin{pmatrix}
        f_1 & \ldots & f_n
    \end{pmatrix}^T \cdot
    \begin{pmatrix}
        f_1 & \ldots & f_n
    \end{pmatrix} = 
    \begin{pmatrix}
        e_1 & \ldots & e_n
    \end{pmatrix}^T \cdot \underbrace{C^TC}_{{} = E}
    \begin{pmatrix}
        e_1 & \ldots & e_n
    \end{pmatrix} = E.
    $$
    Схожим образом доказываются остальные пункты.
\end{proof}


