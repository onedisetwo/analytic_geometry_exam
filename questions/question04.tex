\section{Ортогональные (сохраняющие ориентацию) замены координат на плоскости}

\begin{statement}
    Любая ортогональная матрица $2 \times 2$ с определителем, равным $1$, представляется в виде
    $$
    \begin{pmatrix}
        \cos\varphi & -\sin\varphi\\
        \sin\varphi & \cos\varphi
    \end{pmatrix}\quad\text{и}\quad
    \begin{pmatrix}
        \cos\varphi & \sin\varphi\\
        \sin\varphi & -\cos\varphi
    \end{pmatrix}.
    $$
\end{statement}

\begin{proof}
    Рассмотрим ортогональную матрицу как матрицу перехода от базиса $e_1, e_2$ к $e_1^\prime, e_2^\prime$. Тогда для некоторого $\varphi$ $e_1 = (\cos\varphi, \sin\varphi)$ (т.\,к. $|e_1^\prime| = 1$). Перпендикулярный к нему вектор $e_2^\prime$ единичной длины имеет один из двух видов --- $(-\sin\varphi, \cos\varphi)$ или $(\sin\varphi, -\cos\varphi)$ (все координаты в первом базисе). Располагая эти векторы по столбцам в матрицу, получим ровно два вида --- те, что в формулировке утверждения.
\end{proof}

\begin{orangebox}
    Первая матрица соответствует повороту на угол $\varphi$ против часовой стрелки, а вторая --- симметрии относительно прямой, проходящей через начало координат под углом $\varphi / 2$ к оси абсцисс.
\end{orangebox}

