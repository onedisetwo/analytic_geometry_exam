\section{Гипербола и её геометрические свойства}

\begin{definition}[Геометрическое определение гиперболы]
    \textbf{Гипербола} --- геометрическое место точек $X$, модуль разности от которых до некоторых фиксированных точек $F_1$ и $F_2$ постоянен и меньше $|F_1F_2|$:
    $$
    \big||XF_1| - |XF_2|\big| = 2a.
    $$
    Точки $F_1$ и $F_2$ называются \textbf{фокусами} гиперболы.
\end{definition}

\begin{statement}
    Аналитическое и геометрическое определения гиперболы эквивалентны.
\end{statement}

\begin{theorem}[Оптическое свойство]
    Касательная в точке $X$ гиперболы является биссектрисой угла $\angle F_1XF_2$.
\end{theorem}

\begin{theorem}[Изогональное свойство]
    Пусть дана точка $P$ снаружи гиперболы. Проведём касательные $PA$ и $PB$ к эллипсу из неё. Тогда $PF_1$ и $PF_2$ изогональны относительно угла $\angle APB$.
\end{theorem}


