\section{Деление отрезка в заданном отношении}

\begin{definition}
    Пусть $A$ и $B$ --- различные точки, а $\lambda, \mu \in \R$ не равны нулю одновременно. Говорят, что точка $M$ \textbf{делит отрезок $AB$ в заданном отношении $\lambda \vcentcolon \mu$}, если $\mu\overrightarrow{AM} = \lambda{\overrightarrow{MB}}$.
\end{definition}

\begin{statement}
    В любой аффинной системе координат $\displaystyle M = \frac{\mu A + \lambda B}{\lambda + \mu}$.
\end{statement}

\begin{proof}
    Из определения, $\mu(M - A) = \lambda(B - M)$, отсюда $\displaystyle M = \frac{\mu A + \lambda B}{\lambda + \mu}$.
\end{proof}

\noindent\begin{minipage}{.6\textwidth}
Обычно запись $\lambda \vcentcolon \mu$ заменяют на одно число $\lambda$ (говоря, что точка делит отрезок в отношении $\lambda$), при этом имея в виду, что эта точка делит отрезок в отношении $\lambda \vcentcolon 1$. Такая замена некорректна при $\mu = 0$. Но в этом случае будем считать $\lambda = \infty$. Следующая схема показывает значение $\lambda$ в зависимости от расположения точки $M$:

\begin{orangebox}
    Середина отрезка $AB$ делит его в отношении $1$, точка $A$ --- в отношении $0$, $B$ --- $\infty$. В отношении $-1$ отрезок $AB$ делит бесконечно удалённая точка. Это можно увидеть как из определения, так и из формулы выше.
\end{orangebox}
\end{minipage}
\begin{minipage}{.4\textwidth}
    \centering
    \begin{asy}
        defaultpen(fontsize(11pt));
        usepackage("amsmath");
        usepackage("amssymb");
        settings.tex="lualatex";
        settings.outformat="pdf";

        import geometry;

        size(6cm);
        pair S = (0, 0), F = (9, 6);
        draw(S -- F);
        pair A = (2 * S + F) / 3, B = (S + 2 * F) / 3;
        draw(scale(.75) * Label("$\lambda > 0$", Rotate(dir(S -- F))), align=1.5 * LeftSide, A -- B);
        draw(scale(.75) * Label("$-1 < \lambda < 0$", Rotate(dir(S -- F))), align=1.5 * LeftSide, S -- A);
        draw(scale(.75) * Label("$\lambda < -1$", Rotate(dir(S -- F))), align=1.5 * LeftSide, B -- F);

        dot("$A$", A, dir(-45));
        dot("$B$", B, dir(-45));
    \end{asy}
\end{minipage}

