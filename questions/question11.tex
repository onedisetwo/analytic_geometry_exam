\section{Прямая в пространстве. Параметрическое задание и задание уравнениями}

Прямую в пространстве всё так же можно задать параметрически, выбрав на ней репер:

$$
\begin{cases}
    x = x_0 + \alpha t,\\
    y = y_0 + \beta t,\\
    z = z_0 + \gamma t
\end{cases}\quad t \in \R.
$$

Можно системой уравнений первого порядка:

$$
\begin{cases}
    A_1x + B_1y + C_1z + D_1 = 0,\\
    A_2x + B_2y + C_2z + D_2 = 0
\end{cases}\quad 
$$
где $(A_1, B_1, C_1) \ne (0, 0, 0)$ и $(A_2, B_2, C_2) \ne (0, 0, 0)$ и векторы $(A_1, B_1, C_1, D_1)$ и $(A_2, B_2, C_2, D_2)$ линейно независимы.

А можно в канонической форме:

$$
\frac{x - x_0}{\alpha} = \frac{y - y_0}{\beta} = \frac{z - z_0}{\gamma}
$$

Здесь опять же деление понимается как пропорциональность.

\begin{statement}
    Эти способы задания эквивалентны.
\end{statement}

\begin{proof}
    Для начала, разберёмся с системой. Мы уже знаем, что уравнение первого порядка в пространстве (с выписанными условиями) задаёт плоскость. А система из двух таких уравнений --- пересечение этих плоскостей.

    Пусть имеем параметрическое задание прямой. Выберем ещё два вектора, с которыми направляющий вектор прямой образует линейно независимую систему (можно, т.\,к. размерность пространства 3, а любая линейно независимую систему, в частности, из одного вектора, дополняется до базиса). Если репер прямой дополнить одним из этих векторов, получим одну плоскость, а если другим --- другую. Запишем уравнения этих плоскостей в виде уравнений первого порядка, их пересечение задаётся системой уравнений первого порядка, а это и есть требуемый вид.

    Теперь обратно, пусть имеем систему уравнений первого порядка с указанными выше условиями на коэффициенты. Тогда каждое из уравнений задаёт плоскость, причём они не параллельны и не совпадают. Пусть наши плоскости $\pi_i: A_ix + B_iy + C_iz + D_i = 0$, $i = 1, 2$. Среди $A_1, B_1, C_1$ есть ненулевой. Не ограничивая общности, пусть это $A_1$. Тогда $\displaystyle x = -\frac{D_1}{A_1} - \frac{B_1}{A_1}y - \frac{C_1}{A_1}z$. Подставив в уравнение второй плоскости, получим:
    $$
    A_2\left(-\frac{D_1}{A_1} - \frac{B_1}{A_1}y - \frac{C_1}{A_1}z\right) + B_2y + C_2z + D_2 = 0
    $$
    По сути, мы записали уравнение пересечения этих плоскостей в репере второй плоскости. А получили уравнение первой степени (причём, все коэффициенты линейной части этого уравнения не могут быть нулевыми), то есть, прямую на этой плоскости.

    Равносильность третьего и первого заданий очевидна.
\end{proof}


